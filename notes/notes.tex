\documentclass{article}

\usepackage[a4paper, total={6in, 8in}]{geometry}

\usepackage[T1]{fontenc}
\usepackage[utf8]{inputenc}

\usepackage{amsmath}
\usepackage{braket}
\usepackage{mathtools}

% \title{Cryptography and Security Protocols\\ Project Report}
\title{Shor's Algorithm}
% \author{
%   José Semedo\\
%   ist178294
%   \and
%   Rodrigo Bernardo\\
%   ist178942
% }

\date{}

\begin{document}
\maketitle

% \section{Shor's Algorithm}

\begin{enumerate}
\item Pick $x$ at random s.t. $1<x<N$

\item If $GCD(x,N) = d \neq 1$ then $d$ is a factor of $N$, so we recurse on $N/d$.
Else $x$ and $N$ are coprime, so we try to find the multiplicative order of $x$
modulo $N$.

The quantum computer is initialized to $\ket{\psi_0} = \ket{0}\ket{0}$.
Register one has $t$ qubits ($N^2 \leq 2^t < 2N^2$), and register two has $n =
\lceil{\log _2 N}\rceil$ qubits.

\item Apply the Hadamard operator $t$ times to the first register, yielding
  \begin{equation}
    \ket{\psi_1} = H^{\otimes{t}}\ket{\psi_0}
    = \frac{1}{\sqrt{2^t}}\sum_{j=0}^{2^t-1}\ket{j}\ket{0}
    \,.
  \end{equation}

\item Apply the linear operator $V_x(\ket{j}\ket{k}) = \ket{j}\ket{k+x^j}$ to obtain 

  \begin{equation}
    \begin{aligned}
      \ket{\psi_2} &= V_x\ket{\psi_1}\\
      &= \frac{1}{\sqrt{2^t}}\sum_{j=0}^{2^{t}-1}\ket{j}\ket{x^j}\\
      &= \frac{1}{\sqrt{2^t}}\sum_{b=0}^{r-1}\sum_{a=0}^{\frac{2^t}{r}-1}\ket{ar + b}\ket{x^b}
      \,.
    \end{aligned}
  \end{equation}

\item Measure the second register, fixing $b = b_0$, where $b_0$ is a random
  number between $0$ and $r-1$, obtaining
  \begin{equation}
    \ket{\psi_3}
    = \sqrt{\frac{r}{2^t}}\sum_{a=0}^{\frac{2^t}{r}-1}\ket{ar +
      b_0}\ket{x^{b_0}}
    \,.
  \end{equation}

\item Apply the inverse Fourier transform, $DFT^{\dagger}(\ket{k}) =
  \frac{1}{\sqrt{N}}\sum_{j=0}^{N-1}e^{- 2\pi i jk/N}\ket{j}$, yielding
  \begin{equation}
    \ket{\psi_4}
    = \frac{1}{\sqrt{r}}\left(\sum_{j=0}^{2^t-1}
      \left[\frac{1}{2^{t}/r}\sum_{a=0}^{\frac{2^t}{r}-1}e^{\frac{- 2\pi ija}{2^{t}/r}}\right]
      e^{- 2\pi ijb_{0}/2^t}\ket{j}\right)\ket{x^{b_0}}
    \,.
  \end{equation}

\item Measuring the first register, we get the value $k_{0}2^{t}/r$, for some
  $k_{0} \in \{ 0, 1, ..., r-1 \}$. If we obtain $k_{0} = 0$ we run the
  algorithm again. Else we divide $k_{0}2^{t}/r$ by $2^{t}$, obtaining $k_{0}/r$.

\item To extract $r$, we represent $k_{0}/r$ by a finite continued fraction,
  and then ... ???

\end{enumerate}

\end{document}



























