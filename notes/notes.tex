\documentclass{article}

\usepackage[a4paper, total={6in, 8in}]{geometry}

\usepackage[T1]{fontenc}
\usepackage[utf8]{inputenc}

\usepackage{amsmath}
\usepackage{amsfonts}
\usepackage{braket}
\usepackage{listings}
\usepackage{mathtools}

\title{Simulation of Shor's Algorithm Report}
\author{
  José Semedo\\
  ist178294
  \and
  Rodrigo Bernardo\\
  ist178942
}

\date{}

\begin{document}
\maketitle

\section{Introduction}
In this project we studied and implemented a simulator of Shor's quantum
algorithm for integer factorization. The problem is reduced to the problem of
order finding. Although it is not known if order finding is hard in a classical
setting, Shor \cite{Shor} demonstrated that it is solvable in polynomial time if
one has access to a quantum computer, thus showing integer factorization is
solvable in polynoial time.


\section{Language and Tools}
Our simulation is implemented in Python 3, together with the numpy library.


\section{Implementation Overview}

\subsection{Memory}
The algorithm takes an odd integer $N$, such that it is not a prime nor a power
of a prime, and an integer $x, 1 < x < N$, and tries to find the multiplicative
order of $x$ modulo $N$.

It starts by allocating $t + n$ qubits, with $n = \lceil{\log_2 N}\rceil$ and
$N^2 \leq 2^t < 2N^2$, and initializes the state to $\ket{\psi_{0}} = \ket{0,
0}$.

We represent the memory by explicitly saving all the $2^{t+n}$ possible states and
their corresponding amplitudes, equivalent to the representation of the quantum
state as a linear vector combination $\ket{\psi} =
\sum_{j=0}^{t+n}a_{j}\ket{j}$.

\subsection{Hadamard Gates}
The algorithm then applies Hadamard gates to the first $t$ qubits. This creates
a quantum superposition where the amplitudes are equidistributed between the
first t bits. The state becomes $\ket{\psi_1} = H^{\otimes{t}}\ket{\psi_0} =
2^{-t/2}\sum_{j=0}^{2^t-1}\ket{j}\ket{0}$.

We simulate this step by explicitly reaching for the states where the last $n$
qubits are $\ket{0}$ and setting their amplitudes to $2^{-t/2}$.

\subsection{Modular Exponentiation}

In the next step, the operator $\ket{j, k} \mapsto \ket{j, k + x^j \pmod N}$ is
applied to all the qubits, giving the state
$\ket{\psi_2} =
2^{-t/2}\sum_{j=0}^{2^t-1}\ket{j}\ket{x^b}$.

This step is fast because it generates all the powers
simultaneously by quantum parallelism.

Here we take advantage of Python's built-in modular exponentiation function and
simulate this by applying the operator to all the states sequentially.

\subsection{Quantum Fourier Transform}
The discrete Fourier transform is then applied to the first t qubits. This step
is $O(n2^{n})$ if done classically, but can be done polynomially with a quantum
computer.

We simulate this step by sequentially applying the formula
$\ket{k} \mapsto 2^{-t/2} \sum_{j=0}^{2^{t}-1}\omega^{jk}\ket{j}$, where
$\omega^{jk} = e^{2\pi i j k / N}$,
to all the possible states, i.e., for each state $\ket{\phi} = \ket{k}$, its
amplitude is changed to $2^{-t/2} \sum_{j=0}^{2^{t}-1}\omega^{jk}$.

After the quantum state of the system is
$2^{-t/2}\sum_{j=0}^{2^{t}-1}\sum_{k=0}^{2^{t}-1}\omega^{jk}\ket{k}\ket{x^{j}}$


\subsection{Obtaining the Order}
Finally, a measurement is taken, leaving the state to collapse to one vector of
the computational basis. After the application of the previous operations we are
left with an approximation of a number $a/r, a \in \mathbb{Z}$ with high
probability.

We use a known efficient classical algorithm \cite{HW} for extracting $r$ based
on the best approximation property of the convergents of continued fractions. If
we succeed to find $r$, we return it, else the algorithm is restarted.


\section{Execution}

We can run the program by issuing the command
\begin{lstlisting}
  $ ./shor.py N
\end{lstlisting}

\subsection{Examples}

\begin{lstlisting}
  $ ./shor.py 15
  | picked random a = 3
  | got lucky, 15 = 3 * 5, trying again...
  |---------------------------------------------
  | picked random a = 8
  | measured 59, approximation for 0.23046875 is 3/13
  | 8^13 mod 15 = 8
  | failed, trying again ...
  | measured 246, approximation for 0.9609375 is 1/1
  | 8^1 mod 15 = 8
  | failed, trying again ...
  | measured 109, approximation for 0.42578125 is 3/7
  | 8^7 mod 15 = 2
  | failed, trying again ...
  | measured 222, approximation for 0.8671875 is 7/8
  | 8^8 mod 15 = 1
  | got 8
  | found factor: 15 = 5 * 3
  5
\end{lstlisting}


\begin{lstlisting}
  $ ./shor.py 21
  | picked random a = 10
  | measured 152, approximation for 0.296875 is 3/10
  | 10^10 mod 21 = 4
  | failed, trying again ...
  | measured 342, approximation for 0.66796875 is 2/3
  | 10^3 mod 21 = 13
  | failed, trying again ...
  | measured 37, approximation for 0.072265625 is 1/14
  | 10^14 mod 21 = 16
  | failed, trying again ...
  | measured 53, approximation for 0.103515625 is 2/19
  | 10^19 mod 21 = 10
  | failed, trying again ...
  | measured 42, approximation for 0.08203125 is 1/12
  | 10^12 mod 21 = 1
  | got 12
  | found factor: 21 = 7 * 3
  7 
\end{lstlisting}

\begin{thebibliography}{9}
\bibitem{Shor}
  Peter Shor,
  \emph{Polynomial-Time Algorithms for Prime Factorization and Discrete
Logarithms on a Quantum Computer}

\bibitem{HW}
  G. H. Hardy,
  E. M. Wright,
  \emph{Introduction to Theory of Numbers},
  Oxford University Press,
  4th Edition,
  1975.

\end{thebibliography}

\end{document}




















